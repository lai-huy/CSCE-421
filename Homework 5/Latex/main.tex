\documentclass{article}


% if you need to pass options to natbib, use, e.g.:
%     \PassOptionsToPackage{numbers, compress}{natbib}
% before loading neurips_2022


% ready for submission
\usepackage[final]{neurips_2022}


% to compile a preprint version, e.g., for submission to arXiv, add add the
% [preprint] option:
%     \usepackage[preprint]{neurips_2022}


% to compile a camera-ready version, add the [final] option, e.g.:
%     \usepackage[final]{neurips_2022}


% to avoid loading the natbib package, add option nonatbib:
%    \usepackage[nonatbib]{neurips_2022}


\usepackage[utf8]{inputenc} % allow utf-8 input
\usepackage[T1]{fontenc}    % use 8-bit T1 fonts
\usepackage{hyperref}       % hyperlinks
\usepackage{url}            % simple URL typesetting
\usepackage{booktabs}       % professional-quality tables
\usepackage{amsfonts}       % blackboard math symbols
\usepackage{nicefrac}       % compact symbols for 1/2, etc.
\usepackage{microtype}      % microtypography
\usepackage{xcolor}         % colors
\usepackage{graphicx}

\title{Formatting Instructions For NeurIPS 2022}

% The \author macro works with any number of authors. There are two commands
% used to separate the names and addresses of multiple authors: \And and \AND.
%
% Using \And between authors leaves it to LaTeX to determine where to break the
% lines. Using \AND forces a line break at that point. So, if LaTeX puts 3 of 4
% authors names on the first line, and the last on the second line, try using
% \AND instead of \And before the third author name.


\author{
    Huy Quang Lai \\
    Department of Computer Science and Engineering\\
    Texas A\&M University\\
    College Station, Texas 77843 \\
    \texttt{lai.huy@tamu.edu} \\
}

\begin{document}

\maketitle

\section{Principal Component Analysis}

\subsection{Processing the data and calculating Eigenvalues}
No written part

\subsection{Plot of the Eigenvalues}
The number of components needed to capture 50\% of the energy is $3$. This is displayed in the figure below.
\begin{figure}[!ht]
    \centering
    \includegraphics[width=\textwidth]{eigenvalue.png}
    \caption{Eigenvalue Spectrum}
\end{figure}

\clearpage
\subsection{PCA and Eigen Faces}
\begin{figure}[!ht]
    \centering
    \includegraphics[width=\textwidth]{eigenface.png}
    \caption{10 eigen faces}
\end{figure}

\subsection{Projection and Reconstruction}
\begin{figure}[!ht]
    \centering
    \includegraphics[width=\textwidth]{faces.png}
    \caption{Reconstruction}
\end{figure}

From these visualizations, only one component is required to achieve a visually good result.

\section*{References}
{
\small
[1] Watt, Jeremy, Borhani, Reza \ \& Katsaggelos, Aggelos Konstantinos\ (2016) Machine Learning Refined.

[2] Konasani, Venkata Reddy \ \& Shailendra Kadre\ (2021) Machine Learning and Deep Learning Using Python and TensorFlow.
}
\end{document}
