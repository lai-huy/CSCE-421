\section{Hypothesis Testing}
\subsection{The Hypothesis}
The use of Standard Error allows for hypothesis testing. A very common hypothesis is the Null Hypothesis and its alternative hypothesis.
\begin{align*}
H_0 &:\text{There is no relationship between $x$ and $y$} \\
H_a &:\text{There exists a relationship between $x$ and $y$}
\end{align*}

Mathematically this is equivalent to
\begin{align*}
H_0 &:\beta_1=0, \therefore y=\beta_0+\varepsilon \\
H_a &:\beta_1\neq 0, \therefore \hat{\beta}_1>>0
\end{align*}

\subsection{T-statistic}
\begin{align*}
t_{\beta}&=\frac{\hat{\beta}_1-\beta}{SE\left(\hat{\beta}_1\right)} \\
t_{\beta}&=\frac{\hat{\beta}}{SE\left(\hat{\beta}_1\right)} \text{ for } H_0
\end{align*}

If no relationship between $x$ and $y$ exists, the data is expected to be a t-distribution with P-2 degrees of freedom.
Compute the probability of observing any number equal to $|t|$ or larger in absolute value, assuming $\beta_1=0$.
This probability is called the p-value.
For a small p-value, is unlikely to observe a substantial association between predictor and response due to chance.
Therefore, a small p-value means there is an association between $x$ and $y$ so we can reject the null hypothesis.
The cutoff is usually 5\% or 1\%

\subsection{Example}

\begin{table}[!ht]
\caption{Data for the Example}
\centering
\begin{tabular}{l|llll}
\hline
          & Coefficient & Std. Error & t-statistic & p-value    \\ \hline
Intercept & $7.0325$    & $0.4578$   & $15.36$     & $< 0.0001$ \\
TV        & $0.0475$    & $0.0027$   & $17.67$     & $< 0.0001$ \\ \hline
\end{tabular}
\end{table}

With the sample size of 30, the t-statistic for the null hypothesis of both the Intercept and TV is about $2$ and $2.75$ respectively.
With this result, $\beta_0\neq0$ and $\beta_1\neq1$